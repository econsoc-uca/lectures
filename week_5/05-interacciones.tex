\documentclass[11pt, aspectratio=169, compress]{beamer}
\usetheme[progressbar=frame title, numbering=fraction]{metropolis}      % Use metropolis theme 
\setbeamertemplate{section in toc}[sections numbered]
\setbeamertemplate{subsection in toc}[subsections numbered]
\useoutertheme[subsection=false]{miniframes}
\setbeamercolor{section in head/foot}{fg=white, bg=mDarkTeal}
\setbeamercolor{background canvas}{bg=white}
\setbeamerfont{section in head/foot}{series=\bfseries}

\usefonttheme[onlymath]{serif}
\usepackage{amsmath}
\usepackage{remreset}
\usepackage{ragged2e}
\usepackage{booktabs}
\usepackage{makecell}
\usepackage{float}
\usepackage{subfig}
\usepackage{tikz}
\usetikzlibrary{positioning,calc}
\usepackage[flushleft]{threeparttable}	% 3 part table 
\usepackage[justification=centering]{caption}
\captionsetup{skip=0pt}
\graphicspath{{./fig/}}

\makeatletter
\let\beamer@writeslidentry@miniframeson=\beamer@writeslidentry
\def\beamer@writeslidentry@miniframesoff{%
	\expandafter\beamer@ifempty\expandafter{\beamer@framestartpage}{}% does not happen normally
	{%else
		% removed \addtocontents commands
		\clearpage\beamer@notesactions%
	}
}
\newcommand*{\miniframeson}{\let\beamer@writeslidentry=\beamer@writeslidentry@miniframeson}
\newcommand*{\miniframesoff}{\let\beamer@writeslidentry=\beamer@writeslidentry@miniframesoff}
\beamer@compresstrue
\makeatother

%==============================================================
% Title Page
%==============================================================
%Information to be included in the title page:
\title{Interacciones sociales | Economía del comportamiento}
\author{Rony Rodriguez-Ramírez} 
\institute{Economía Social y Humana | Grupo B018 \\Universidad Centroamericana}
\titlegraphic{\hfill\includegraphics[height=1.5cm]{uca}}
\date{\today}
%==============================================================
\begin{document}
	
\begin{frame}[plain]
	\maketitle  
\end{frame}

%\begin{frame}{Outline}
%\tableofcontents[hideallsubsections]
%\end{frame}
%------------------------------------------------
\section{Comentarios y anuncios}
%-----------------------------------------------
\subsection{Comentarios y anuncios}
%-----------------------------------------------
\begin{frame}{Comentario y anuncios}
\begin{itemize}
	\item ¿Cómo les fue en el ensayo? 
	\item Relación instituciones y economía social: 
	\begin{itemize}
		\item ¿Por qué el paper de los esclavos en Africa la semana pasada? 
	\end{itemize}
	\item La economía social como filosofía política vs la economía social como la intervención social en la economía. 
\end{itemize}
\end{frame}
%------------------------------------------------
\section{Introducción a la economía del comportamiento}
%------------------------------------------------
\subsection{Introducción a la economía del comportamiento}
%------------------------------------------------
\begin{frame}{Economía del comportamiento}
De manera general: 

\end{frame}
%------------------------------------------------
\begin{frame}{Ejemplos de los sistemas}
\begin{enumerate}
	\item Un bate y una pelota cuestan 1,10 dólares en total. El bate cuesta 1 dólar
	más que la pelota. ¿Cuánto cuesta la pelota? \_\_\_ centavos.
	\item Si 5 máquinas hacen 5 artículos en 5 minutos, ¿cuánto tardarán 100
	máquinas en hacer 100 artículos? \_\_\_ minutos.
	\item En un lago hay una superficie cubierta de nenúfares. Cada día esa extensión
	dobla su tamaño. Si tarda 48 días en cubrir todo el lago, ¿cuánto tarda en
	cubrir la mitad del lago? \_\_\_ días.
\end{enumerate}
\end{frame}
%------------------------------------------------
\begin{frame}{Reglas básicas}
	Para Thaler y Sunstein (2017) las reglas básicas son muy útiles pero nos pueden conducir a \textit{sesgos}. Tres reglas básicas: 
	\begin{itemize}
		\item Anclaje. 
		\item Disponibilidad. 
		\item Represntatividad.  
	\end{itemize}
\end{frame}
%------------------------------------------------
\begin{frame}{Reglas básicas}
	Anclaje: 
	\begin{itemize}
		\item Se comienza con una \textit{ancla} y se ajusta a la dirección que se considera apropriada. Ejemplo de la población. 
		\item Dos tipos de anclas: altas y bajas. 
		\item Las anclas actuan como \textit{nudges} (empujones).  
	\end{itemize}
\end{frame}
%------------------------------------------------
\begin{frame}{Disponibilidad}
	La disponibilidad se trata de evaluar el riesgo asociado a ciertos ejemplos. 
	\begin{itemize}
		\item Si alguno de ustedes ha experimentado cierto evento (o program), seguramente tendrán mejor noción sobre la importancia o la magnitud de ese evento. 
		\item El heurístico de la disponibilidad explica las conductas relacionadas al riesgo. 
		\item Valoraciones sesgadas: 
		\begin{itemize}
			\item influyen en cómo nos preparamos para las decisiones económicas, crisis, etc. 
			\item Valoración real del riesgo actual de un evento. 
		\end{itemize}
	\end{itemize}
\end{frame}
%------------------------------------------------
\begin{frame}{Representatividad}
	Representatividad o Semejanza: 
	\begin{itemize}
		\item Este es un sesgo con bastante frecuencia. 
		\item Puede causar distorsiones en la percepción. 
		\begin{itemize}
			\item Ejemplo de la tirar la moneda al aire y que caiga tres veces cara.
			\item Ejemplo de la muñeca caliente en el baloncesto. 
		\end{itemize}
		\item 
	\end{itemize}
\end{frame}
%------------------------------------------------
\begin{frame}{Optimismo y exceso de confianza}
	Emprender un negocio y el dilema de la esperanza: 
	\begin{itemize}
		\item El optimismo no realista puede en cierta medida explicar los riesgos individuales. 
		\item Es un rasgo universal de la vida humana. 
		\item Siempre se puede utilizar un nudge para balancear este optimismo. 
	\end{itemize}
\end{frame}
%------------------------------------------------
\begin{frame}{Algunos otros ejemplos}
	\begin{itemize}
		\item Gananzas y pérdidas. 
		\item El sesgo del statu quo.		
	\end{itemize}
\end{frame}
%------------------------------------------------
\section{Relación con la economía social}
\subsection{Relación con la economía social}
%------------------------------------------------
\begin{frame}{Panorama general}
	Aunque pareciera, de manera general, que no hay relación, si la hay: 
	\begin{itemize}
		\item Las personas adoptan reglas básicas que influyen en sus decisiones. 
		\item Muchas veces reaccionamos de manera muy sesgada. 
		\item A las personas se les pueden orientar mediante \textit{nudges}.
	\end{itemize}
\end{frame}
%------------------------------------------------
\begin{frame}{Economía social y economía del comportamiento}
	Retomémoms un poco sobre la probabilidad de existo un program social, una cooperativa, entre otras cosas. 
	\begin{itemize}
		\item ¿Qué sesgos se les viene a la mente? 
		\item ¿Por qué fracasarían? ¿Por qué tendrían exito? 
	\end{itemize}
\end{frame}
%------------------------------------------------
\begin{frame}{Economía social y economía del comportamiento}
Con la economía del comportamiento podemos ver como se han impulsados programas sociales o actividades de apoyo comunitario. 
\begin{itemize}
	\item La próxima semana comenzaremos de lleno a ver ejemplos de papers que han trabajados estos empujones.
\end{itemize}
\end{frame} 

%==============================================================
% END
%==============================================================
\miniframesoff 	
\begin{frame}[plain, standout]
Nos vemos la siguiente semana. 
\end{frame}

\end{document}		
