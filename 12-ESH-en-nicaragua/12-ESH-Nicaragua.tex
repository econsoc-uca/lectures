\documentclass[11pt, aspectratio=169, compress]{beamer}
\usetheme[progressbar=frame title, numbering=fraction]{metropolis}      % Use metropolis theme 
\setbeamertemplate{section in toc}[sections numbered]
\setbeamertemplate{subsection in toc}[subsections numbered]
\useoutertheme[subsection=false]{miniframes}
\setbeamercolor{section in head/foot}{fg=white, bg=mDarkTeal}
\setbeamercolor{background canvas}{bg=white}
\setbeamerfont{section in head/foot}{series=\bfseries}

\usefonttheme[onlymath]{serif}
\usepackage{amsmath}
\usepackage{remreset}
\usepackage{ragged2e}
\usepackage{booktabs}
\usepackage{makecell}
\usepackage{float}
\usepackage{subfig}
\usepackage{tikz}
\usetikzlibrary{positioning,calc}
\usepackage[flushleft]{threeparttable}	% 3 part table 
\usepackage[justification=centering]{caption}
\captionsetup{skip=0pt}
\graphicspath{{./fig/}}

\makeatletter
\let\beamer@writeslidentry@miniframeson=\beamer@writeslidentry
\def\beamer@writeslidentry@miniframesoff{%
	\expandafter\beamer@ifempty\expandafter{\beamer@framestartpage}{}% does not happen normally
	{%else
		% removed \addtocontents commands
		\clearpage\beamer@notesactions%
	}
}
\newcommand*{\miniframeson}{\let\beamer@writeslidentry=\beamer@writeslidentry@miniframeson}
\newcommand*{\miniframesoff}{\let\beamer@writeslidentry=\beamer@writeslidentry@miniframesoff}
\beamer@compresstrue
\makeatother

%==============================================================
% Title Page
%==============================================================
%Information to be included in the title page:
\title{La Economía Social en Nicaragua}
\author{Rony Rodriguez-Ramírez} 
\institute{Economía Social y Humana | Grupo B018 \\Universidad Centroamericana}
\titlegraphic{\hfill\includegraphics[height=1.5cm]{uca}}
\date{\today}
%==============================================================
\begin{document}
	
\begin{frame}[plain]
	\maketitle  
\end{frame}

%\begin{frame}{Outline}
%\tableofcontents[hideallsubsections]
%\end{frame}
%------------------------------------------------
\section{La Economía Social en Nicaragua}
%-----------------------------------------------
\subsection{La Economía Social en Nicaragua}
%-----------------------------------------------
\begin{frame}{Alianza Cooperativa Internacional (2007)}
Lógica de la economía social: 

\begin{itemize}
	\item Se relaciona con lo asociativo, comunitario y todo tipo de organización económica que cumpla con los principios de ayuda mutua y solidaridad. 
	\item Autogestión, control de los trabajadores de las unidades productivas. 
\end{itemize}

Dinámica del sector social en Nicaragua:
\begin{itemize}
	\item ¿Por qué es pobre Nicaragua? 
\end{itemize}

\end{frame}
%-----------------------------------------------
\begin{frame}{Dinámica del sector social en Nicaragua}
	Pobreza en Nicaragua:
	\begin{itemize}
		\item Se ha mantenido un sistema generador de desigualdades y pobreza.
		\item La desigualdad económica y social ha venido aumentando desde los años noventa.
		\item Remesas y cuentapropia. 
		\item Recursos naturales y desechos. 
		\item Problemas fiscales.
	\end{itemize}
\end{frame}
%-----------------------------------------------
\begin{frame}{Dinámica del sector social en Nicaragua}
	Pobreza en Nicaragua:
	\begin{itemize}
		\item Bono demográfico y declive de fertilidad.
		\item Condiciones precarias del empleo.
		\item Desigualdades de género. 
		\item Discriminación laboral. 
	\end{itemize}
\end{frame}
%------------------------------------------------
\begin{frame}{Evolución del sector de la economía social}
	Evolución del sector cooperativo: 
	\begin{itemize}
		\item Década de los 30s. 
		\item Alianza para el progreso: Federación de Cooperativas de Ahorro y Crédito (FECANIC). 
		\item Periodo revolucionario: expansión del cooperativismo. ¿Mejoras en la productividad? 
		\item Situación de guerra y deterioro. 
		\item Años noventas: cooperativimso multisectorial o multifuncional. 
		\item Los "Sin Tierra" y los cuentapropistas.
		\item Sector informal.
	\end{itemize}
\end{frame}
%------------------------------------------------
\begin{frame}{Actores principales de la economía social}
	Economía social informal: 
	\begin{itemize}
		\item Cuentapropismo en el sector informal urbano: vendedores ambulantes, cambistas, etc.
		\item Poco apoyo del Estado.
	\end{itemize}
	Economía social asociativa:
	\begin{itemize}
		\item Colectivos productivos, cooperativas, empresas comunales, entre otros. 
		\item No sólo responden a las necesidades económicas sino de otros tipos, como: culturales, políticas, educativa.
	\end{itemize}
\end{frame}
%------------------------------------------------
\section{Políticas públicas para la economía social}
%-----------------------------------------------
\subsection{Políticas públicas para la economía social}
%-----------------------------------------------
\begin{frame}{Políticas públicas para una economía social y solidaria}
Enfoque 1990 - 2006: 
\begin{itemize}
	\item Respuesta a la vulnerabilidad, desempoderamiento, e injusticia. 
	\item Apoyo de ONGs. 
\end{itemize}
Periodo 2007 - Actualidad: 
\begin{itemize}
	\item Cambio institucional: enfoque a la micro, pequeñas, y medianas empresas. 
	\item Promoción de crédito a través de microfinancieras. 
	\item Incremente de programas sociales. 
\end{itemize}
\end{frame}
%-----------------------------------------------
\begin{frame}{Política social y del mercado laboral}
Salud, educación, y mercado laboral:
\begin{itemize}
	\item Educación pública. 
	\item Merienda diaria. 
	\item Programas sociales financiados a través del ALBA. 
	\item Creciente papel del Instituto Nacional Tecnológico. 
\end{itemize}	
\end{frame}
%-----------------------------------------------
\begin{frame}{Programas sociales}
Crecimiento de los programas sociales:
\begin{itemize}
	\item ¿A qué se debe este incrementeo de programas sociales? 
\end{itemize}
\end{frame}
%-----------------------------------------------
\begin{frame}{Programas sociales: Hambre cero}
\begin{itemize}
	\item Promover seguridad alimentaria y empoderamiento de las mujeres en zonas rurales y periurbanas. 
\end{itemize}
\end{frame}

%-----------------------------------------------
%==============================================================
% END
%==============================================================
\miniframesoff 	
\begin{frame}[plain, standout]
¡Esto fue todo! 
\end{frame}
%------------------------------------------------
\end{document}		
